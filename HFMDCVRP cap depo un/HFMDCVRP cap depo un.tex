% This is file `elsarticle-template-1a-num.tex',
%
% Copyright 2009 Elsevier Ltd.
%
% This file is part of the 'Elsarticle Bundle'.
% ---------------------------------------------
%
% It may be distributed under the conditions of the LaTeX Project Public
% License, either version 1.2 of this license or (at your option) any
% later version.  The latest version of this license is in
%    http://www.latex-project.org/lppl.txt
% and version 1.2 or later is part of all distributions of LaTeX
% version 1999/12/01 or later.
%
% The list of all files belonging to the 'Elsarticle Bundle' is
% given in the file `manifest.txt'.
%
% Template article for Elsevier's document class `elsarticle'
% with numbered style bibliographic references
%
% $Id: elsarticle-template-1a-num.tex 151 2009-10-08 05:18:25Z rishi $
% $URL: http://lenova.river-valley.com/svn/elsbst/trunk/elsarticle-template-1a-num.tex $
%
%\documentclass[12pt]{elsarticle}

% Use the option review to obtain double line spacing
 \documentclass[preprint,review,12pt]{elsarticle}

% Use the options 1p,twocolumn; 3p; 3p,twocolumn; 5p; or 5p,twocolumn
% for a journal layout:
% \documentclass[final,1p,times]{elsarticle}
% \documentclass[final,1p,times,twocolumn]{elsarticle}
% \documentclass[final,3p,times]{elsarticle}
% \documentclass[final,3p,times,twocolumn]{elsarticle}
% \documentclass[final,5p,times]{elsarticle}
% \documentclass[final,5p,times,twocolumn]{elsarticle}

% if you use PostScript figures in your article
% use the graphics package for simple commands
% \usepackage{graphics}
% or use the graphicx package for more complicated commands
% \usepackage{graphicx}
% or use the epsfig package if you prefer to use the old commands
% \usepackage{epsfig}

% The amssymb package provides various useful mathematical symbols
\usepackage{amssymb}
% The amsthm package provides extended theorem environments
% \usepackage{amsthm}

% The lineno packages adds line numbers. Start line numbering with
% \begin{linenumbers}, end it with \end{linenumbers}. Or switch it on
% for the whole article with \linenumbers after \end{frontmatter}.
% \usepackage{lineno}

% natbib.sty is loaded by default. However, natbib options can be
% provided with \biboptions{...} command. Following options are
% valid:
\usepackage{natbib}
%   round  -  round parentheses are used (default)
%   square -  square brackets are used   [option]
%   curly  -  curly braces are used      {option}
%   angle  -  angle brackets are used    <option>
%   semicolon  -  multiple citations separated by semi-colon
%   colon  - same as semicolon, an earlier confusion
%   comma  -  separated by comma
%   numbers-  selects numerical citations
%   super  -  numerical citations as superscripts
%   sort   -  sorts multiple citations according to order in ref. list
%   sort&compress   -  like sort, but also compresses numerical citations
%   compress - compresses without sorting
%
% \biboptions{comma,round}

\usepackage{amsmath}
%\usepackage[latin1]{inputenc}
\usepackage[utf8]{inputenc}
\usepackage{algorithm2e}
\usepackage{algorithmicx}
\usepackage{algpseudocode}
%\usepackage{pstricks,pst-node} %Removed by Diana. It causes me an error when I added the adjustbox package
\usepackage{multirow}
\usepackage{multicol}
\usepackage{rotating}
%\usepackage{tabularx}
%\usepackage{tabulary}
%\usepackage{pstricks-add}
\usepackage{booktabs}
%\usepackage{color}
%\usepackage{algorithm}
%\usepackage{algpseudocode}

\newcommand\fede[1]{\textbf{{\color{red}#1}}}
\newcommand\juan[1]{\textbf{{\color{blue}#1}}}
 \biboptions{}
\newtheorem{proposition}{Proposition}[section]
\newtheorem{lemma}{Lemma}[section]
\newtheorem{theorem}{Theorem}[section]
\newtheorem{example}{Example}[section]
\newtheorem{remark}{Remark}[section]
\newtheorem{definition}{Definition}[section]
\newtheorem{corollary}{Corollary}[section]
\def\proof{\noindent {\bf Proof. }}
\def\endproof{{\bf \hfill $\square$}}


\usepackage{fixltx2e}
\usepackage{float}
\usepackage{adjustbox}
\usepackage{xcolor}

\begin{document}

\section{HFMDCVRP formulation}

\subsection{Introduction}
This formulation is for a HFMDCVRP (Heterogeneous Fleet Multi Depot Capacitated Vehicle Routing Problem) where the fleet is composed by different vehicles and more than one depot can dispatch the vehicles. The objective is simple: assign a route and a depot to every vehicle  to attend the demand of all the clients.

\subsection{Sets}
The HFMDCVRP is designed upon a connected and directed Graph $G = (N,A)$. The set of nodes $N$ is composed by a set of clients $C$ and a set of depots $D$ ($N = C \cup D$). The set of arcs is represented by $A$. The set $K$ corresponds to the set of available vehicles to attend the demand. As the fleet is supposed to be heterogeneous, the capacity of each vehicle $k$ is given by $q_k$. The cost of using the arc $(i,j)$ is represented with $c_{ij}^k$ for vehicle $k$. The demand of every client is $dem_c$. Lastly, as the depots have a finite capacity to attend the vehicles, a maximum depot capacity $R_d$ is added to each depot $d$ (this capacity is given in terms of the demand the depot can attend).

\subsection{Variables}
It uses the following sets of variables:

\begin{itemize}
	\item $x_{ij}^k \in \{0,1\}$: binary variable which takes the value 1 if the arc $(i,j)$ is used by the $k^{th}$ vehicle, and zero otherwise. The number of variables of this family is $|K| \cdot |N^2|$.
	\item $f_{ij}^k \geq 0$: continuous auxiliary variable which represents the demand already attended when vehicle $k$ use arc $(i,j)$. The objective of this variable is to ensure that the routes of each vehicle are well-defined and avoid cycling. It is also important to evaluate the capacity of the depots. The number of variables of this family is $|K| \cdot |N^2|$.
	\item  $w_d^k \in  \{0,1\}$: Binary variable which takes the value 1 if vehicle $k$ is served from depot $d$ and zero otherwise. The number of variables of this family is $|K| \cdot |D|$.
	\item $z_{c}^k \in  \{0,1\}$: Binary variable which takes the value 1 if vehicle $k$ serves client $c$ and zero otherwise. The number of variables of this family is $|K| \cdot |C|$.
\end{itemize}

\subsection{Formulation}

\begin{align}
    \text{(HFMDCVRP) min} \quad        & Z = \sum\limits_{k\in K}\sum\limits_{i\in N}\sum\limits_{j\in N} c_{ij}^k \cdot x_{ij}^k    && \label{MDMVRP_FObj}    \\[5pt]
	%\noindent \text{subject to} \quad & \sum\limits_{k \in K}\sum\limits_{j \in N} x_{ij}^k = 1  && \text{\hspace{-1cm}} \forall \ i \in C \label{MDMVRP_Service} \\
	\noindent \text{subject to} \quad & \sum\limits_{k \in K} z_{c}^k = 1  && \text{\hspace{-1cm}} \forall \ c \in C \label{MDMVRP_Service} \\
	& \sum\limits_{c \in C}dem_c \cdot z_{c}^k \leq q_k               && \text{\hspace{-1cm}} \forall \ k \in K  \label{MDMVRP_Capacity} \\
	& \sum\limits_{j \in N}x_{dj}^k = w_d^k                                         && \text{\hspace{-1cm}} \forall \ d \in D,\ k \in K \label{MDMVRP_StartRoute} \\
	& \sum\limits_{i \in N}x_{id}^k = w_d^k                                         && \text{\hspace{-1cm}} \forall \ d \in D,\ k \in K \label{MDMVRP_EndRoute} \\
	& \sum\limits_{d \in D} w_d^k \leq 1 && \text{\hspace{-1cm}} \forall \ k \in K \label{MDMVRP_OneDepo} \\
	& \sum\limits_{i \in N}x_{ic}^k - \sum\limits_{j \in N}x_{cj}^k = 0             && \text{\hspace{-1cm}} \forall \ c \in C,\ k \in K \label{MDMVRP_Balance} \\
	& \sum\limits_{d \in D}\sum\limits_{c \in C}f_{dc}^k = 0 && \text{\hspace{-1cm}} \forall \ k \in K \label{MDVMRP_InitialF} \\
	& \sum\limits_{i \in N}f_{ic}^k - \sum\limits_{j \in N}f_{cj}^k + dem_c \cdot z_{c}^k = 0             && \text{\hspace{0cm}} \forall \ c \in C,\ k \in K \label{MDMVRP_Flow} \\
	& \sum\limits_{i \in N}\sum\limits_{k \in K}f_{id}^k \leq R_d  && \text{\hspace{-1cm}} \forall \ d \in D \label{MDMVRP_DepotQ} \\
	& w_d^k \in\{0,1\}  && \text{\hspace{-1cm}} \forall \ d \in D, \ k \in K \label{MDVMRP_DomainW} \\
	& x_{ij}^k\in\{0,1\} && \text{\hspace{-1cm}} \forall \ i,j \in N,  \ k \in K \label{MDVMRP_DomainX} \\
	& 0 \leq f_{ij}^k \leq q_k \cdot x_{ij}^k && \text{\hspace{-1cm}} \forall \ i,j \in N,  \ k \in K \label{MDVMRP_DomainF} \\
	& z_{c}^k\in\{0,1\} && \text{\hspace{-1cm}} \forall \ c \in C,  \ k \in K \label{MDVMRP_DomainZ} 
\end{align}

Constraints \eqref{MDMVRP_Service} ensure that every client is visited at least once. Inequalities \eqref{MDMVRP_Capacity} limit the amount of clients served by a vehicle according to their capacity. Constraints \eqref{MDMVRP_StartRoute} and \eqref{MDMVRP_EndRoute} guarantee that the vehicles depart and finish the route at the depot assigned to them. Inequality \eqref{MDMVRP_OneDepo} only allows vehicles to be attended from one depot. Equalities \eqref{MDMVRP_Balance} assure that the flow continues through the network. Equation \eqref{MDVMRP_InitialF} determines that all the initial arcs have not attended any demand. Equations \eqref{MDMVRP_Flow} adds the demand attended in every client to the next arc used. Inequalities \eqref{MDMVRP_DepotQ} guarantee that the capacity of every depot is not exceeded. The last four expressions define the domain of the decision variables.

\end{document}

