% This is file `elsarticle-template-1a-num.tex',
%
% Copyright 2009 Elsevier Ltd.
%
% This file is part of the 'Elsarticle Bundle'.
% ---------------------------------------------
%
% It may be distributed under the conditions of the LaTeX Project Public
% License, either version 1.2 of this license or (at your option) any
% later version.  The latest version of this license is in
%    http://www.latex-project.org/lppl.txt
% and version 1.2 or later is part of all distributions of LaTeX
% version 1999/12/01 or later.
%
% The list of all files belonging to the 'Elsarticle Bundle' is
% given in the file `manifest.txt'.
%
% Template article for Elsevier's document class `elsarticle'
% with numbered style bibliographic references
%
% $Id: elsarticle-template-1a-num.tex 151 2009-10-08 05:18:25Z rishi $
% $URL: http://lenova.river-valley.com/svn/elsbst/trunk/elsarticle-template-1a-num.tex $
%
%\documentclass[12pt]{elsarticle}

% Use the option review to obtain double line spacing
 \documentclass[preprint,review,12pt]{elsarticle}

% Use the options 1p,twocolumn; 3p; 3p,twocolumn; 5p; or 5p,twocolumn
% for a journal layout:
% \documentclass[final,1p,times]{elsarticle}
% \documentclass[final,1p,times,twocolumn]{elsarticle}
% \documentclass[final,3p,times]{elsarticle}
% \documentclass[final,3p,times,twocolumn]{elsarticle}
% \documentclass[final,5p,times]{elsarticle}
% \documentclass[final,5p,times,twocolumn]{elsarticle}

% if you use PostScript figures in your article
% use the graphics package for simple commands
% \usepackage{graphics}
% or use the graphicx package for more complicated commands
% \usepackage{graphicx}
% or use the epsfig package if you prefer to use the old commands
% \usepackage{epsfig}

% The amssymb package provides various useful mathematical symbols
\usepackage{amssymb}
% The amsthm package provides extended theorem environments
% \usepackage{amsthm}

% The lineno packages adds line numbers. Start line numbering with
% \begin{linenumbers}, end it with \end{linenumbers}. Or switch it on
% for the whole article with \linenumbers after \end{frontmatter}.
% \usepackage{lineno}

% natbib.sty is loaded by default. However, natbib options can be
% provided with \biboptions{...} command. Following options are
% valid:
\usepackage{natbib}
%   round  -  round parentheses are used (default)
%   square -  square brackets are used   [option]
%   curly  -  curly braces are used      {option}
%   angle  -  angle brackets are used    <option>
%   semicolon  -  multiple citations separated by semi-colon
%   colon  - same as semicolon, an earlier confusion
%   comma  -  separated by comma
%   numbers-  selects numerical citations
%   super  -  numerical citations as superscripts
%   sort   -  sorts multiple citations according to order in ref. list
%   sort&compress   -  like sort, but also compresses numerical citations
%   compress - compresses without sorting
%
% \biboptions{comma,round}

\usepackage{amsmath}
%\usepackage[latin1]{inputenc}
\usepackage[utf8]{inputenc}
\usepackage{algorithm2e}
\usepackage{algorithmicx}
\usepackage{algpseudocode}
%\usepackage{pstricks,pst-node} %Removed by Diana. It causes me an error when I added the adjustbox package
\usepackage{multirow}
\usepackage{multicol}
\usepackage{rotating}
%\usepackage{tabularx}
%\usepackage{tabulary}
%\usepackage{pstricks-add}
\usepackage{booktabs}
%\usepackage{color}
%\usepackage{algorithm}
%\usepackage{algpseudocode}

\newcommand\fede[1]{\textbf{{\color{red}#1}}}
\newcommand\juan[1]{\textbf{{\color{blue}#1}}}
 \biboptions{}
\newtheorem{proposition}{Proposition}[section]
\newtheorem{lemma}{Lemma}[section]
\newtheorem{theorem}{Theorem}[section]
\newtheorem{example}{Example}[section]
\newtheorem{remark}{Remark}[section]
\newtheorem{definition}{Definition}[section]
\newtheorem{corollary}{Corollary}[section]
\def\proof{\noindent {\bf Proof. }}
\def\endproof{{\bf \hfill $\square$}}


\usepackage{fixltx2e}
\usepackage{float}
\usepackage{adjustbox}
\usepackage{xcolor}

\begin{document}

\section{PCVRP formulation}

\subsection{Sets}
The PCVRP is designed upon a connected and directed Graph $G = (N,A)$ where $N$ is the set of nodes and $A$ is the set of arcs. The set of clients $C$ and a depot $0$, $(N=C \cup \{0\})$ are also given. The cost of using the arc $(i,j)$ is represented by $c_{ij}$ (because the fleet is homogeneous, the cost is the same for all the vehicles). The set of days is $\Delta$. The index set of available vehicles to serve the demand is $K$ and the demand of every client is $d_{i\delta}$. Lastly, as it is supposed a homogeneous fleet, the capacity, $q$, is the same for all vehicles.

\subsection{Variables}
It uses the following sets of variables:

\begin{itemize}
	\item  $x_{ij}^{k\delta}  \in  \{ 0,1 \}$: binary variable which takes the value 1 if the arc $(i,j)$ is used by the $k^{th}$ vehicle in the day $\delta$, and zero otherwise. The number of variables of this family is $|K||\Delta|\cdot|N^2|$.
	\item $u_{i\delta} \geq 0$: continuous auxiliary variable which represents the position in which node $i$ is visited on its route in day $\delta$. The objective of this variable is to ensure that the routes of each vehicle are well-defined and avoid cycling. The number of variables of this family is $|N||\Delta|$.
\end{itemize}

\subsection{Formulation}

\begin{align}
	%
	\text{(VRP) min} \quad      & Z = \sum\limits_{\delta\in \Delta} \sum\limits_{k\in K}\sum\limits_{i\in N}\sum\limits_{j\in N} c_{ij} \cdot x_{ij}^{k\delta}    && \label{PCVRP_FObj}    \\[5pt]
	%
	\noindent \text{subject to} \quad & \sum\limits_{k \in K}\sum\limits_{j \in N} x_{ij}^{k\delta} = 1                    && \text{\hspace{-1cm}} \forall \ i \in C, \forall \ \delta \in \Delta \label{PCVRP_Service} \\
	%
	& \sum\limits_{i \in C}\sum\limits_{j \in N}d_{i\delta} \cdot x_{ij}^{k\delta} \leq q                 && \text{\hspace{-1cm}} \forall \ k \in K, \forall \ \delta \in \Delta  \label{PCVRP_Capacity} \\
	& \sum\limits_{j \in N}x_{0j}^{k\delta} = 1                                             && \text{\hspace{-1cm}} \forall \ k \in K, \forall \ \delta \in \Delta \label{PCVRP_StartRoute} \\
	& \sum\limits_{i \in N}x_{i0}^{k\delta} = 1                                             && \text{\hspace{-1cm}} \forall \ k \in K, \forall \ \delta \in \Delta \label{PCVRP_EndRoute} \\
	& \sum\limits_{i \in N}x_{ih}^{k\delta} - \sum\limits_{j \in N}x_{hj}^{k\delta} = 0             && \text{\hspace{-1cm}} \forall \ h \in C,\ k \in K, \forall \ \delta \in \Delta \label{PCVRP_Balance} \\
	& u_{0\delta} = 1                                                                       && \text{\hspace{-1cm}} \forall \ \delta \in \Delta \label{PCVRP_InitialU}\\
	& u_{j\delta} \geq (u_{i\delta} + 1) - M \left(1 - \sum\limits_{k \in K}x_{ij}^{k\delta} \right) \qquad && \text{\hspace{-1cm}} \forall \ i \in N,\ j \in C, \forall \ \delta \in \Delta \label{PCVRP_OrderU}\\
	& x_{ij}^{k\delta} \in\{0,1\}                                                         && \text{\hspace{-1cm}} \forall \ i, \ j \in N,\ k \in K,\ \delta \in \Delta \label{PCVRP_DomainX}\\
	& 2 \leq u_{i\delta} \leq |N|                                                           && \text{\hspace{-1cm}} \forall \ i \in C, \forall \ \delta \in \Delta \label{PCVRP_DomainU}
S\end{align}

Constraints \eqref{PCVRP_Service} ensure that every client is visited at least once everyday. Inequalities \eqref{PCVRP_Capacity} limit the amount of clients served by a vehicle according to their capacity everyday. Constraints \eqref{PCVRP_StartRoute} and \eqref{PCVRP_EndRoute} guarantee that the vehicles depart and finish the route at the depot 0 always. Equalities \eqref{PCVRP_Balance} assure that the flow continues through the network. Equation \eqref{PCVRP_InitialU} is used to secure that the first node to be visited according to auxiliary variables $u$ is the depot everyday. Inequalities \eqref{PCVRP_OrderU} determine that the value of $u_{j\delta}$ must be higher than $u_{i\delta}$ everyday, when $i$ is the previously visited node. These two last sets of constraints avoid the possibility of having subcircuits in the routes. The last two expressions define the domain of the decision variables.

\end{document}

