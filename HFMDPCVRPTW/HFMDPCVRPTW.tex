% This is file `elsarticle-template-1a-num.tex',
%
% Copyright 2009 Elsevier Ltd.
%
% This file is part of the 'Elsarticle Bundle'.
% ---------------------------------------------
%
% It may be distributed under the conditions of the LaTeX Project Public
% License, either version 1.2 of this license or (at your option) any
% later version.  The latest version of this license is in
%    http://www.latex-project.org/lppl.txt
% and version 1.2 or later is part of all distributions of LaTeX
% version 1999/12/01 or later.
%
% The list of all files belonging to the 'Elsarticle Bundle' is
% given in the file `manifest.txt'.
%
% Template article for Elsevier's document class `elsarticle'
% with numbered style bibliographic references
%
% $Id: elsarticle-template-1a-num.tex 151 2009-10-08 05:18:25Z rishi $
% $URL: http://lenova.river-valley.com/svn/elsbst/trunk/elsarticle-template-1a-num.tex $
%
%\documentclass[12pt]{elsarticle}

% Use the option review to obtain double line spacing
 \documentclass[preprint,review,12pt]{elsarticle}

% Use the options 1p,twocolumn; 3p; 3p,twocolumn; 5p; or 5p,twocolumn
% for a journal layout:
% \documentclass[final,1p,times]{elsarticle}
% \documentclass[final,1p,times,twocolumn]{elsarticle}
% \documentclass[final,3p,times]{elsarticle}
% \documentclass[final,3p,times,twocolumn]{elsarticle}
% \documentclass[final,5p,times]{elsarticle}
% \documentclass[final,5p,times,twocolumn]{elsarticle}

% if you use PostScript figures in your article
% use the graphics package for simple commands
% \usepackage{graphics}
% or use the graphicx package for more complicated commands
% \usepackage{graphicx}
% or use the epsfig package if you prefer to use the old commands
% \usepackage{epsfig}

% The amssymb package provides various useful mathematical symbols
\usepackage{amssymb}
% The amsthm package provides extended theorem environments
% \usepackage{amsthm}

% The lineno packages adds line numbers. Start line numbering with
% \begin{linenumbers}, end it with \end{linenumbers}. Or switch it on
% for the whole article with \linenumbers after \end{frontmatter}.
% \usepackage{lineno}

% natbib.sty is loaded by default. However, natbib options can be
% provided with \biboptions{...} command. Following options are
% valid:
\usepackage{natbib}
%   round  -  round parentheses are used (default)
%   square -  square brackets are used   [option]
%   curly  -  curly braces are used      {option}
%   angle  -  angle brackets are used    <option>
%   semicolon  -  multiple citations separated by semi-colon
%   colon  - same as semicolon, an earlier confusion
%   comma  -  separated by comma
%   numbers-  selects numerical citations
%   super  -  numerical citations as superscripts
%   sort   -  sorts multiple citations according to order in ref. list
%   sort&compress   -  like sort, but also compresses numerical citations
%   compress - compresses without sorting
%
% \biboptions{comma,round}

\usepackage{amsmath}
%\usepackage[latin1]{inputenc}
\usepackage[utf8]{inputenc}
\usepackage{algorithm2e}
\usepackage{algorithmicx}
\usepackage{algpseudocode}
%\usepackage{pstricks,pst-node} %Removed by Diana. It causes me an error when I added the adjustbox package
\usepackage{multirow}
\usepackage{multicol}
\usepackage{rotating}
%\usepackage{tabularx}
%\usepackage{tabulary}
%\usepackage{pstricks-add}
\usepackage{booktabs}
%\usepackage{color}
%\usepackage{algorithm}
%\usepackage{algpseudocode}

\newcommand\fede[1]{\textbf{{\color{red}#1}}}
\newcommand\juan[1]{\textbf{{\color{blue}#1}}}
 \biboptions{}
\newtheorem{proposition}{Proposition}[section]
\newtheorem{lemma}{Lemma}[section]
\newtheorem{theorem}{Theorem}[section]
\newtheorem{example}{Example}[section]
\newtheorem{remark}{Remark}[section]
\newtheorem{definition}{Definition}[section]
\newtheorem{corollary}{Corollary}[section]
\def\proof{\noindent {\bf Proof. }}
\def\endproof{{\bf \hfill $\square$}}


\usepackage{fixltx2e}
\usepackage{float}
\usepackage{adjustbox}
\usepackage{xcolor}

\begin{document}

\section{HFMDPCVRPTW formulation}

\subsection{Introduction}
This formulation is a PCVRP (Periodic Capacitated Vehicle Routing Problem) where the fleet of vehicles is heterogeneous and can be served from different depots. Additionally the clients must be served within a time-window. The clients have a defined frequency of visits in the time span:

\begin{itemize}
	\item T6: client must be visited everyday.
	\item T3: clients must be visited three times per week.
	\item T2: clients must be visited two times per week.
	\item T1: clients must be visited once per week.
	\item Q1: clients must be visited once every two weeks.
\end{itemize}

\subsection{Sets and parameters}
\begin{itemize}
	\item Set of all the nodes $N$. Subindexes $i,j$ will be used to indicate a element of $N$.
	\item $A$ are the arcs formed between all nodes.
	\item Directed Graph $G = (N,A)$.
	\item Set of clients $C$. Subindex $c$ will be used to indicate a element of $C$.
	\item Set of depots $D$. Subindex $d$ will be used to indicate a element of $D$.
	\item Set of days  $\Delta$. Subindex $\delta$ will be used to indicate a element of $\Delta$.
	\item Set of vehicles $K$. Subindex $k$ will be used to indicate a element of $K$.
	\item Set of patterns of visits available $P$. Subindex $p$ will be used to indicate a element of $P$.
	\item $F_{p\delta}$ is 1 if the pattern $p$ requires the client to be visited in the day $\delta$ and 0 otherwise.
	\item $H_{cp}$ is 1 if client $c$ is eligible for pattern $p$ and 0 otherwise.
	\item The capacity of vehicle $k$ is $q_k$.
	\item The demand of client $c$ for each visit is $dem_c$.
	\item The time to serve client $c$ is $s_c$.
	\item The window of time in which client $c$ must be attended is given by $[a_c,b_c]$.
	\item The time to traverse arc $(i,j)$ with vehicle $k$ is given by $t_{ij}^k$.
	\item The capacity to attend vehicles in depot $d$ is $R_d$.
	\item A largely enough value (\textbf{M}) is used. It can be bounded by maximum time of a route.
	
	
\end{itemize}
 
 %The set of nodes $N$ is composed by a set of clients $C$ and a set of depots $D$ ($N = C \cup D$). The cost of using the arc $(i,j)$ is represented by $c_{ij}$ (because the fleet is homogeneous, the cost is the same for all the vehicles). The set of days is $\Delta$. The index set of available vehicles to serve the demand is $K$ and the demand of every client is $d_{c\delta}$. Lastly, as it is supposed a homogeneous fleet, the capacity, $q$, is the same for all vehicles. A largely enough value (\textbf{M}) is used. It can be bounded by the number of clients.

\subsection{Variables}
It uses the following sets of variables:

\begin{itemize}
	\item  $x_{ij}^{k\delta}  \in  \{ 0,1 \}$: binary variable which takes the value 1 if the arc $(i,j)$ is used by the $k^{th}$ vehicle in the day $\delta$, and 0 otherwise. 
	\item $T_{i\delta} \geq 0$: time at which service starts in node $i$ in the day $\delta$. 
	\item $y_{cp} \in  \{ 0,1 \}$: binary variable which takes the value 1 if client $c$ is visited according to pattern $p$ and 0 otherwise.
	\item  $w_d^k \in  \{0,1\}$: Binary variable which takes the value 1 if vehicle $k$ is served from depot $d$ and zero otherwise.
\end{itemize}

\subsection{Formulation}

\begin{align}
	%
	\text{(VRP) min} \quad      & Z = \sum\limits_{\delta\in \Delta} \sum\limits_{k\in K}\sum\limits_{i\in N}\sum\limits_{j\in N} c_{ij}^k \cdot x_{ij}^{k\delta}    && \label{PCVRP_FObj}    \\[5pt]
	%
	\noindent \text{s.t.} \quad & \sum\limits_{k \in K}\sum\limits_{j \in N} x_{cj}^{k\delta} = \sum\limits_{p \in P} F_{p\delta} \cdot y_{cp}    && \text{\hspace{-0cm}} \forall \ c \in C, \forall \ \delta \in \Delta \label{PCVRP_Service} \\
	%
	& \sum\limits_{p \in P} H_{cp} \cdot y_{cp} = 1                 && \text{\hspace{-1cm}} \forall \ c \in C  \label{PCVRP_Freq} \\
	& \sum\limits_{c \in C}\sum\limits_{j \in N}dem_{c} \cdot x_{cj}^{k\delta} \leq q_k                 && \text{\hspace{-1cm}} \forall \ k \in K, \forall \ \delta \in \Delta  \label{PCVRP_Capacity} \\
	& \sum\limits_{j \in N}x_{dj}^{k\delta} \leq w_d^k && \text{\hspace{-1cm}} \forall \ k \in K, \forall \ \delta \in \Delta, \forall \ d \in D \label{PCVRP_StartRoute} \\
	& \sum\limits_{i \in N}x_{id}^{k\delta} =    \sum\limits_{j \in N}x_{dj}^{k\delta} && \text{\hspace{-1cm}} \forall \ k \in K, \forall \ \delta \in \Delta, \forall \ d \in D \label{PCVRP_EndRoute} \\
	& \sum\limits_{i \in N}x_{ic}^{k\delta} - \sum\limits_{j \in N}x_{cj}^{k\delta} = 0             && \text{\hspace{-1cm}} \forall \ c \in C,\ \forall k \in K,\ \forall \ \delta \in \Delta \label{PCVRP_Balance} \\
	& \sum\limits_{k \in K}w_d^k \leq R_d                                     && \text{\hspace{-1cm}} \forall \ d \in D \label{PCVRP_CapacityD} \\
	& T_{d\delta} = 0  && \text{\hspace{-1cm}} \forall \ \delta \in \Delta, \ \forall d \in D \label{PCVRP_InitialT}\\
	& T_{c\delta} \geq a_c  && \text{\hspace{-1cm}} \forall \ \delta \in \Delta, \ \forall c \in C \label{PCVRP_ArriveT}\\
	& T_{c\delta} \leq b_c - s_c && \text{\hspace{-1cm}} \forall \ \delta \in \Delta, \ \forall c \in C \label{PCVRP_DepartT}
\end{align}

\begin{align}
	& T_{c\delta} \geq (T_{i\delta} + \sum\limits_{k \in K} t_{ic}^k \cdot  x_{ic}^{k\delta} + s_i) - M \left(1 - \sum\limits_{k \in K}x_{ic}^{k\delta} \right) \qquad && \text{\hspace{-1cm}} \forall \ i \in N,\ \forall c \in C, \forall \ \delta \in \Delta \label{PCVRP_OrderT}\\
	& x_{ij}^{k\delta} \in\{0,1\} && \text{\hspace{-1cm}} \forall \ i, \ j \in N,\ k \in K,\ \delta \in \Delta \label{PCVRP_DomainX}\\
	& 0 \leq T_{c\delta} \leq M && \text{\hspace{-1cm}} \forall \ c \in C, \forall \ \delta \in \Delta \label{PCVRP_DomainT}\\
	& y_{cp} \in\{0,1\} && \text{\hspace{-1cm}} \forall \ c \in C,\ \forall p \in P \label{PCVRP_DomainY}\\
	& w_d^k \in\{0,1\} && \text{\hspace{-1cm}} \forall \ d \in D,\ \forall k \in K \label{PCVRP_DomainW}
\end{align}

Constraints \eqref{PCVRP_Service} ensure that every client is visited according to the frequency selected for that client. Equalities \eqref{PCVRP_Freq} guarantee that a pattern is selected for every client. Inequalities \eqref{PCVRP_Capacity} limit the amount of clients served by a vehicle according to their capacity everyday. Constraints \eqref{PCVRP_StartRoute} and \eqref{PCVRP_EndRoute} guarantee that the vehicles can only depart and finish the route at the depot it is assigned. Equalities \eqref{PCVRP_Balance} assure that the flow continues through the network. Inequalities \eqref{PCVRP_CapacityD} keep the capacity of the depots at check. Equation \eqref{PCVRP_InitialT} is used to secure that the time of departure from the depot is 0. Constraints \eqref{PCVRP_ArriveT} and \eqref{PCVRP_DepartT} forces the time of arrival of the routes to be inside the time windows. Inequalities \eqref{PCVRP_OrderT} determine that, when $i$ is the previously visited node, the time of arrival to the new node is greater than te time needed to get from $c$ to $i$ plus the time expended in $i$. The last two expressions define the domain of the decision variables.

\end{document}

