% This is file `elsarticle-template-1a-num.tex',
%
% Copyright 2009 Elsevier Ltd.
%
% This file is part of the 'Elsarticle Bundle'.
% ---------------------------------------------
%
% It may be distributed under the conditions of the LaTeX Project Public
% License, either version 1.2 of this license or (at your option) any
% later version.  The latest version of this license is in
%    http://www.latex-project.org/lppl.txt
% and version 1.2 or later is part of all distributions of LaTeX
% version 1999/12/01 or later.
%
% The list of all files belonging to the 'Elsarticle Bundle' is
% given in the file `manifest.txt'.
%
% Template article for Elsevier's document class `elsarticle'
% with numbered style bibliographic references
%
% $Id: elsarticle-template-1a-num.tex 151 2009-10-08 05:18:25Z rishi $
% $URL: http://lenova.river-valley.com/svn/elsbst/trunk/elsarticle-template-1a-num.tex $
%
%\documentclass[12pt]{elsarticle}

% Use the option review to obtain double line spacing
 \documentclass[preprint,review,12pt]{elsarticle}

% Use the options 1p,twocolumn; 3p; 3p,twocolumn; 5p; or 5p,twocolumn
% for a journal layout:
% \documentclass[final,1p,times]{elsarticle}
% \documentclass[final,1p,times,twocolumn]{elsarticle}
% \documentclass[final,3p,times]{elsarticle}
% \documentclass[final,3p,times,twocolumn]{elsarticle}
% \documentclass[final,5p,times]{elsarticle}
% \documentclass[final,5p,times,twocolumn]{elsarticle}

% if you use PostScript figures in your article
% use the graphics package for simple commands
% \usepackage{graphics}
% or use the graphicx package for more complicated commands
% \usepackage{graphicx}
% or use the epsfig package if you prefer to use the old commands
% \usepackage{epsfig}

% The amssymb package provides various useful mathematical symbols
\usepackage{amssymb}
% The amsthm package provides extended theorem environments
% \usepackage{amsthm}

% The lineno packages adds line numbers. Start line numbering with
% \begin{linenumbers}, end it with \end{linenumbers}. Or switch it on
% for the whole article with \linenumbers after \end{frontmatter}.
% \usepackage{lineno}

% natbib.sty is loaded by default. However, natbib options can be
% provided with \biboptions{...} command. Following options are
% valid:
\usepackage{natbib}
%   round  -  round parentheses are used (default)
%   square -  square brackets are used   [option]
%   curly  -  curly braces are used      {option}
%   angle  -  angle brackets are used    <option>
%   semicolon  -  multiple citations separated by semi-colon
%   colon  - same as semicolon, an earlier confusion
%   comma  -  separated by comma
%   numbers-  selects numerical citations
%   super  -  numerical citations as superscripts
%   sort   -  sorts multiple citations according to order in ref. list
%   sort&compress   -  like sort, but also compresses numerical citations
%   compress - compresses without sorting
%
% \biboptions{comma,round}

\usepackage{amsmath}
%\usepackage[latin1]{inputenc}
\usepackage[utf8]{inputenc}
\usepackage{algorithm2e}
\usepackage{algorithmicx}
\usepackage{algpseudocode}
%\usepackage{pstricks,pst-node} %Removed by Diana. It causes me an error when I added the adjustbox package
\usepackage{multirow}
\usepackage{multicol}
\usepackage{rotating}
%\usepackage{tabularx}
%\usepackage{tabulary}
%\usepackage{pstricks-add}
\usepackage{booktabs}
%\usepackage{color}
%\usepackage{algorithm}
%\usepackage{algpseudocode}

\newcommand\fede[1]{\textbf{{\color{red}#1}}}
\newcommand\juan[1]{\textbf{{\color{blue}#1}}}
 \biboptions{}
\newtheorem{proposition}{Proposition}[section]
\newtheorem{lemma}{Lemma}[section]
\newtheorem{theorem}{Theorem}[section]
\newtheorem{example}{Example}[section]
\newtheorem{remark}{Remark}[section]
\newtheorem{definition}{Definition}[section]
\newtheorem{corollary}{Corollary}[section]
\def\proof{\noindent {\bf Proof. }}
\def\endproof{{\bf \hfill $\square$}}


\usepackage{fixltx2e}
\usepackage{float}
\usepackage{adjustbox}
\usepackage{xcolor}

\begin{document}

\section{CVRP formulation integrating in one matrix all the vehicles}

\subsection{Sets}
The CVRP is designed upon a connected and directed Graph $G = (N,A)$ where $N$ is the set of nodes and $A$ is the set of arcs. The set of clients $C$ and a depot $0$, $(N=C \cup \{0\})$ are also given. The cost of using the arc $(i,j)$ is represented by $c_{ij}$ (because the fleet is homogeneous, the cost is the same for all the vehicles). The number of vehicles available is $Vehic$ and the demand of every client is $d_i$. Lastly, as it is supposed a homogeneous fleet, the capacity, $q$, is the same for all vehicles.

\subsection{Variables}
It uses the following sets of variables:

\begin{itemize}
	\item  $x_{ij}  \in  \{ 0,1 \}$: binary variable which takes the value 1 if the arc $(i,j)$ is used by some vehicle.
	\item $f_{ij}$: represents the cumulative packages delivered of the vehicle in every node. Because a matrix will carry all the information, this variable is needed to control the capacity of the vehicles. This variable also guarantees that there will not be any subcircuit.
\end{itemize}

\subsection{Formulation}

\begin{align}
	%
	\text{(VRP) min} \quad      & Z = \sum\limits_{i\in N}\sum\limits_{j\in N} c_{ij} \cdot x_{ij}    && \label{VRP_FObj}    \\[5pt]
	%
	\noindent \text{subject to} \quad & \sum\limits_{j \in N} x_{ij} = 1                    && \text{\hspace{-1cm}} \forall \ i \in C \label{VRP_Service} \\
	%
	& \sum\limits_{j \in N} x_{0j} \leq Vehic                    && \text{\hspace{-1cm}}  \label{VRP_NumVehic} \\
	%
	& \sum\limits_{i \in N}x_{i0} =  \sum\limits_{j \in N}x_{0j}                                             && \text{\hspace{-1cm}}  \label{VRP_EndRoute} \\
	%
	& \sum\limits_{i \in N}x_{ih} - \sum\limits_{j \in N}x_{hj} = 0             && \text{\hspace{-1cm}} \forall \ h \in C \label{VRP_Balance} \\
	%
	& \sum\limits_{j \in N}f_{0j}  = 0             && \text{\hspace{-1cm}} \label{VRP_IniFlux} \\
	%
	& \sum\limits_{j \in N}f_{ij} - \sum\limits_{j \in N}f_{ji} - d_i  = 0             && \text{\hspace{-1cm}} \forall \ i \in C \label{VRP_Flux} \\
	%
	& x_{ii} = 0             && \text{\hspace{-1cm}} \forall \ i \in N \label{VRP_Diag} \\
	%
	& x_{ij} \in\{0,1\}                                                         && \text{\hspace{-1cm}} \forall \ i, \ j \in N, \label{VRP_DomainX}\\
	%
	& 0 \leq f_{ij} \leq q*x_{ij}                                                           && \text{\hspace{-1cm}} \forall \ i \in N, j \in N \label{VRP_DomainF}
\end{align}

Constraints \eqref{VRP_Service} ensure that every client is visited at least once. Constraint \eqref{VRP_NumVehic} guarantees that the maximum number of vehicles used is the number of vehicles available. Equality \eqref{VRP_EndRoute} guarantee that the same number of vehicles that departed from the depot return to it. Equalities \eqref{VRP_Balance} assure that the flow continues through the network. Equation \eqref{VRP_IniFlux} makes the initial flow of every vehicle equal to zero and \eqref{VRP_Flux} adds the demand of every node to the flow.  The last two expressions define the domain of the decision variables.

\end{document}

